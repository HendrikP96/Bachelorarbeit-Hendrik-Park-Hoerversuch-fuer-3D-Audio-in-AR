\chapter{Einleitung}
Laserschwertkämpfe, ein Besuch auf dem Mars oder die Fähigkeit ohne Flugzeug fliegen zu können sind Bestrebungen des Menschen, die entweder noch in ferner Zukunft liegen oder wohl niemals Realität werden. Doch auch wenn der einfache Mensch diese Erfahrungen wohl niemals wirklich erleben wird, so ist der Schritt diese Erlebnisse nachzuempfinden mit der heutigen Technik gar kein so großer mehr. Der Fortschritt in der virtuellen Realität lässt uns einen simulierten Mars besuchen, die Entwicklung von Augmented lässt in der realen Welt eine Fernbedienung als Laserschwert erscheinen  und auch das Fliegen durch simulierte Welten kann so echt wirken, dass manch ein Mensch mit der Höhenangst zu kämpfen hat.\\

Um diese virtuellen Erfahrungen so plausibel wie möglich zu gestalten ist es wichtig die Inhalte an den Menschen anzupassen. Die Überlagerung der Realität mit virtuellen Inhalten wird als Augmented Reality bezeichnet und während die Anforderungen an  eine visuelle Überlagerungen schon sehr weit erforscht ist, so ist dies für auditive Überlagerung bisher nicht der Fall. 

Durch diverse Software ist es bereits möglich Schallquellen in der virtuellen Realität so zu position, dass es so klingen soll als wären diese tatsächlich an der festgelegten Position im Raum. Dies soll es Entwicklern ermöglichen extrem realistische Akustik in virtuellen Welten zu erzeugen. Wie plausibel die räumliche Wiedergabe funktioniert und wie diese vom Nutzer angenommen wird, soll im Rahmen dieser Arbeit überprüft werden. Hierfür wurde ein Hörversuch entwickelt, bei dem Probanden in einer Mixed Reality- Umgebung virtuelle Schallquellen lokalisieren sollten.

Der durchgeführte Hörversuch stellt einen typischen Anwendungsfall für Augmented Reality dar. Es werden virtuelle Inhalte in die reale Welt projiziert, die währenddessen durch binaurale Hörereignisse ergänzt werden. Für die Probanden gilt es die Position der virtuellen Schallquellen anhand des Schalls zu bestimmen und anschließend die Darbietung anhand bestimmte Kriterien zu bewerten. Die Plausibelität der Hörereignisse und das Immersionsempfinden des Probanden stehen hierbei im Vordergrund.

